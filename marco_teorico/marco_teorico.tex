En este capítulo se describirá el modelo en el cual se basa toda la investigación, también se definirán los conceptos utilizados a lo largo del trabajo como la definición de la presión social y los blocking clusters. Además se comentará el algoritmo de Verlet, ya que fue utilizado para resolver la dinámica del problema. 

\section{Modelo de Fuerza social}

El modelo de fuerza social es un modelo basado en agentes utilizado para simular el movimiento de peatones. Supone que los individuos soportan tres tipos de fuerzas: fuerzas de deseo (autopropulsión), sociales (repulsión) y granulares (rozamiento).  \\

La fuerza de autopropulsión refleja el hecho que el individuo i-esimo que poseé masa $m_i$, desea  moverse con una velocidad $v_d^ {(i)}(t)$ en una dirección $\hat{\mathbf{e}}_d^ {(i)}(t)$, por lo tanto readapta su velocidad $\mathbf{v}_i(t)$ con un cierto tiempo característico $\tau$.
\begin{equation}
\mathbf{f}_d^ {(i)}(t)=m_i\,\displaystyle\frac{v_d^ {(i)}(t)\,\hat{\mathbf{e}}_d^ {(i)}(t)-\mathbf{v}_i(t)}{\tau}\label{fdeseo}
\end{equation}

La repulsión social es una fuerza que describe la tendencia que tienen las personas a mantenerse alejadas unas de otras. Depende de la distancia de separación entre individuos $d_{ij}=\left\|\mathbf{r_i}-\mathbf{r_j}\right\|$ y está en la dirección normal $\mathbf{n}_{ij}=(n_{ij}^1,n_{ij}^2)/d_{ij}$ (versor que apunta desde el individuo j al individuo i). Los peatones están en contacto si el valor de $d_{ij}$ es más chico que la suma de los radios $r_{ij}=(r_i+r_j)$. $A_i$ y 	$B_i$ son constantes.

\begin{equation}
\mathbf{f}_s^{(ij)}=A_i\,e^{(r_{ij}-d_{ij})/B_i}\mathbf{n}_{ij}\label{fsocial}
\end{equation} 

Cuando los individuos están en contacto, comienza a actuar una fuerza de rozamiento, la misma es proporcional a la velocidad relativa entre individuos.  
$\Delta \mathbf{v}_{ij}=(\mathbf{v}_i-\mathbf{v}_j)$, la dirección tangencia está representada por $\mathbf{t}_{ij}=(-n_{ij}^2,n_{ij}^1)$.  $\kappa$ es una constante y $g(x)$ es una función nula cuando los individuos no se toca $(r_{ij}<d_{ij})$ y toma el valor de su argumento en caso contrario. 
\begin{equation}
\mathbf{f}_g^{(ij)}=\kappa\,g(r_{ij}-d_{ij})\,\Delta \mathbf{v}_{ij}\cdot\mathbf{t}_{ij}\label{frozamiento}
\end{equation}

Se trata de forma análoga a la interacción de los individuos con las paredes. $d_{iW}$ es la distancia del i-esimo peatón con la pared W, $n_{iW}$ la dirección perpendicular entre éstos y $t_{iW}$ la tangencial. La expresión \ref{fparedes} agrupa tanto a la repulsión como al rozamiento de la interacción peatón-pared.

\begin{equation}
\mathbf{f}^{iW}=A_ie^{(r_{i}-d_{iW})/B_i}\mathbf{n}_{iW}-\kappa g(r_{i}-d_{iW})\Delta \mathbf{v}_{i}\cdot\mathbf{t}_{iW}
\label{fparedes}
\end{equation} 

Con todo esto, mediante la fórmula \ref{newton}, puede expresarse el cambio de velocidad en el tiempo que siente un individuo. $f^{ij}$ es el término de interacción entre individuos, es la suma de la repulsión y el rozamiento. El cambio de posición viene dado por $v_{i}(t)=d\mathbf{r_i}/dt$.

\begin{equation}
m_i\frac{d\mathbf{v_i}}{dt}=\mathbf{f}_d^ {(i)}(t)+ \sum_{i\neq j}^{N}\mathbf{f}^{(ij)} + \sum_{W}^{N}\mathbf{f}^{(iW)}
\label{newton}
\end{equation}  
 
\section{Presión social}

\section{Blocking Clusters}

Se utilizó el concepto de blocking cluster para estudiar a los individuos más cercanos a la salida. Se lo define como un conjunto de individuos en contacto que va desde un punto de la pared hacia otro. Estos puntos están proximos a los costados de la puerta. Aquellos peatones que conforman el blocking cluster son los responsables de bloquear la salida.  Tipicamente los blocking cluster formados en las evacuaciones tiene estructura de "arco" como el que se muestra en la figura \ref{bc}. Basta que un solo peaton deje de estar en contacto con sus vecinos para que el blocking cluster deje de estar constituído. \\

\begin{figure}[H]
    \centering
    \includegraphics[height=5.5cm]{figuras/dos_puertas.png}
    \caption[width=5cm]{}
    \label{bc}
\end{figure}

A lo largo del trabajo se han usado dos tipos de blocking clusters: Big blocking cluster y small blocking cluster. El primero se tiene cuando hay dos puertas sobre la misma pared, son los individuos que van desde una puerta hasta la otra (encierran las dos salidas). El segundo es el bloqueo de una única salida. 

\section{Algoritmo de Verlet}

Para integrar las ecuaciones de moviemiento se usó el algoritmo de Verlet. El mismo se basa en actualizar las velocidades, luego las posiciones, en tercer lugar actualiza las fuerzas y vuelve a repetir el ciclo hasta completar las trayectorias. 

%% Poner fórmulas, ventajas y orden del error. (ver haile)


