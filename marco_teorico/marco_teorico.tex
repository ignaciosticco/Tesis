En este capítulo se describirá el modelo en el cual se basa toda la investigación, también se definirán los conceptos utilizados a lo largo del trabajo como la definición de la presión social y los blocking clusters. Además se comentará el algoritmo de Verlet, ya que fue utilizado para resolver la dinámica del problema. 

\section{Modelo de Fuerza social}

El modelo de fuerza social es un modelo basado en agentes utilizado para simular el movimiento de peatones. Supone que los individuos soportan tres tipos de fuerzas: fuerzas de deseo (autopropulsión), sociales (repulsión) y granulares (rozamiento).  \\

\begin{equation}
\mathbf{f}_d^ {(i)}(t)=m_i\,\displaystyle\frac{v_d^ {(i)}(t)\,\hat{\mathbf{e}}_d^ {(i)}(t)-\mathbf{v}_i(t)}{\tau}\label{eqn_1review}
\end{equation}

\begin{equation}
\mathbf{f}_s^{(ij)}=A_i\,e^{(r_{ij}-d_{ij})/B_i}\mathbf{n}_{ij}\label{eqn_1}
\end{equation} 

\begin{equation}
\mathbf{f}_g^{(ij)}=\kappa\,g(r_{ij}-d_{ij})\,\Delta \mathbf{v}_{ij}\cdot\mathbf{t}_{ij}
\end{equation}
