
\noindent En este capítulo se describirán los métodos usados para llevar a cabo las simulaciones. 

\noindent Para realizar las simulaciones se utilizó el programa LAMMPS (Large-scale Atomic/Molecular Massively Parallel Simulator).
Es un software de código libre distribuido bajo los términos de GPL.
LAMMPS se caracteriza por hacer uso de las listas de vecinos [rapaport] para efecuar cálculos que permiten reducir la
complejidad algorítmica, además cuenta con una gran cantidad de funciones implementadas orientadas al uso de simulaciones
de dinámica molecular. 

\noindent Todas las simulaciones constaron de N individuos en un recinto cuadrado cuyo tamaño estaba ligado a la cantidad de
peatones de modo tal que mantenga constante la densidad. En una de las paredes se ubicó una o dos puertas dependiendo de qué 
se buscaba analizar.
Para todas los sistemas estudiados se configuró un arreglo bidimensional de individuos, ordenados inicialmente tipo
red cuadrada, separados entre si por una distancia de 1,3 m. La velocidad inicial se estipuló de modo que todos los individuos
tengan en promedio 1.7 m/s (en módulo) con una dispersion de ## m/s, la dirección fue generada aleatoriamente para
cada uno de ellos. Luego del instante inicial, los agentes cambiaban su velocidad acorde a la velocidad de deseo configurada 
(con el fin de que todos busquen evacuar la habitación). Una vez que los individuos abandonaban el recinto no se los reinyectaba.
De modo que al evacuar dejaba de importar sus observables. 
Para resolver la dinámica se utilizó el algoritmo de Verlet. 
\noindent Con el fin de compatibilizar el modelo de fuerza social con LAMMPS, se crearon varios módulos con las fuerzas que
caracterizan al modelo.   







