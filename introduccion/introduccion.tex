\section{Antecedentes}

A lo largo de los últimos años se evidenció cómo la dinámica descontrolada de multitudes (\emph{i.e.} en estado de pánico) llevó a tener que lamentar numerosas víctimas. Se recuerdan las fatalidades  de  la “puerta 12” en el estadio de River Plate (1968) y la catástrofe en el salón de espectáculos de Cromagnon (2004), por mencionar algunos hechos conmocionantes. Otras fatalidades se han convertido en recurrentes como las del puente de Jamaraat, durante la peregrinación anual a la  Meca. \\   

\noindent Estos sucesos ocasionados por el movimiento de multitudes en estado de pánico ha atraído la atención de investigadores en diversas áreas (psicología social, ingeniería, física, etc.). Diversos modelos han sido propuestos para describir esta problemática, que van desde la dinámica de fluidos hasta los autómatas celulares.\\

\noindent En una aproximación al continuo, algunos autores utilizan ecuaciones de evolución del tipo fluido.  El esfuerzo está puesto en analizar la evolución temporal de una cierta “densidad de masa de individuos” y del  flujo, que viene a representar, de manera simplificada, el movimiento de personas~\cite{bruno}. \\

\noindent Por otro lado, los modelos de tipo autómata celular asimilan a los individuos como agentes móviles sobre un arreglo regular de “celdas”. Éstos se desplazan “celda a celda” de acuerdo a reglas diseñadas para reproducir comportamientos típicos de peatones en multitudes. Tienen la ventaja de ser muy rápidos y fácilmente entendibles para los no habituados a problemas de dinámica molecular o granular. Sin embargo, no son muy aptos para describir la emergencia de situaciones de alta densidad y las correspondientes variaciones de la dinámica~\cite{blue}.\\

\noindent Una mejor aproximación al estudio del movimiento de multitudes es el modelo de Fuerza Social de Helbing~\cite{Helbing1}. El modelo de Helbing se basa en observaciones empíricas sobre el comportamiento social de individuos. Considera que el movimiento de individuos se debe a una combinación de fuerzas socio-psicológicas y físicas. Las fuerzas ingresan entonces a una ecuación de movimiento clásica. \\

\noindent El modelo básico de Helbing identifica al menos tres tipos de fuerzas. La "fuerza de deseo", las “fuerzas sociales”,  y una fuerza granular.  La primera corresponde al “deseo” del individuo por moverse en cierta dirección (como consecuencia de sus propias motivaciones), la segunda está asociada a la “preservación” de su “espacio de privacidad”, o sea, la tendencia de los individuos a estar alejados unos de otros. La fuerza granular es la  fricción experimentada entre individuos en contacto. En las Refs~\cite{Helbing1,Dorso1,Dorso3} se puede encontrar información detallada sobre cada una de las fuerzas.\\ 

\noindent La “fuerza de deseo” es el resultado de las motivaciones de los individuos. Allí se incluye, no sólo el nivel de ansiedad de una persona para llegar a destino (vía la velocidad a la que "quiere" moverse el individuo), sino también otros factores de su personalidad y cultura, como por ejemplo, la tendencia a seguir a otros, o de establecer consensos~\cite{Dorso3,Dorso4}.  Puede consultarse las Refs.~\cite{Wang,low} para algunas situaciones específicas. 
La formulación matemática del modelo de Helbing será detallada en el capítulo \ref{marco_teo}. \\ 

El modelo de fuerza social es capaz de describir varios fenómenos propios de las evacuaciones en estado de pánico. Entre ellos el efecto más sobresaliente es el "más rápido es más lento" (faster is slower)~\cite{Helbing1}. Cuando los individuos se esfuerzan más por salir, más tiempo demoran. En este régimen, se producen demoras debido al incremento en la fricción entre personas y a la formación de estrucutras bloqueantes ("blocking cluters"). En las Refs. \cite{Dorso1} se pueden encontrar más detalles al respecto.

\noindent El modelo básico de Helbing cubre una amplia variedad de situaciones y puede utilizarse para modelar evacuaciones en recintos con geometrías complejas. En este trabajo de tesis se aborda la problemática de evacuaciones en estado de pánico en recintos con una y dos puertas sobre la misma pared.\\ 

\noindent La práctica de incluir dos puertas para evacuaciones de emergencia se remonta a los tiempos de la última dinastía Qing en China (1644-1911 AD). Una norma legal estableció que los edificios grandes debían tener dos salidas para evitar problemas en caso de incendio~\cite{cheng}.\\

Las regulaciones edilicias actuales (de Estados Unidos) cuentan con especificaciones bien detalladas sobre la ubicación de las puertas de emergencia, su ancho y la separación entre éstas~\cite{OSHA,FLO}. Además exigen que el ancho mínimo de las puertas debe ser 0,813~m y el tamaño máximo de una hoja no debe ser mayor a 1,219~m\cite{FLO,FLO2}. Si se requieren más de dos puertas, la separación entre dos de ellas debe ser al menos un cuarto o un tercio de la distancia diagonal máxima de la habitación. No hay ningún requerimiento especial sobre el resto de las puertas, más allá del hecho que no deben estar simultáneamente bloqueadas\cite{FLO,FLO2}.\\

\noindent La regulación vigente permite que cualquier salida adicional a las dos obligatorias sea ubicada en una posición arbitraria. De esta forma, es posible encontrar configuraciones de dos puertas en un mismo lado del recinto (existiendo otra a una distancia mayor). El caso especial de dos puertas contiguas ha sido examinado en la literatura~\cite{kirchner1,perez1,daoliang1,huanhuan1}. \\

\noindent Kirchner y Schadschneider estudiaron el proceso de evacuación de peatones a través de dos puertas contiguas usando un modelo de autómatas celulares\cite{kirchner1}. Los agentes podían abandonar la habitación siguiendo comportamientos que iban desde el individualismo hasta un movimiento fuertemente acoplado como el de una \emph{manada}. Según este modelo, el tiempo de evacuación resultó ser independiente de la distancia entre las puertas contiguas (cuando los individuos se comportan de manera individualista). Pero, si los peatones se movían en manada, se reportó un tiempo de evacuación mayor   para separaciones entre puertas (contiguas) menores a 10 individuos.\\

\noindent Otras investigaciones parecen no concordar con las observaciones de Kirchner y Schadschneider. En la Ref.~\cite{perez1} se concluye que el número de peatones que abandona el recinto por unidad de tiempo mostró una disminución para distancias de separación menores al ancho de cuatro puertas. Esta disminución del flujo fue asociada a un efecto de interferencia debido a peatones cruzándose mutuamente. El umbral de cuatro anchos de puerta ($4\,d_w$) corresponde a la distancia de separación necesaria para distinguir dos  grupos independientes de peatones, cada uno de ellos rodeando a la puerta más cercana (en recintos con 200 individuos). \\

\noindent Los investigadores destacaron el hecho de que para distancias de separación muy grandes entre puertas, el rendimiento de la evacuación no mejora más del doble respecto al rendimiento que tiene una única puerta (con ancho simple). Este efecto se le atribuye a algún tipo de interferencia entre peatones\cite{perez1}.\\

\noindent Aunque los resultados arriba mencionados fueron obtenidos para puertas muy angostas (\emph{i.e.} del ancho de un individuo),
investigaciones posteriores exhiben resultados similares para anchos de puertas equivalentes a dos individuos. A pesar de ello, si los peatones salen por una única salida, el flujo de salida depende del ancho de la puerta, de acuerdo a los resultados mostrados en la Ref.~\cite{daoliang1}. La tasa del flujo dividido el ancho de la puerta (es decir, flujo/ancho) parece decrecer aunque el flujo total aumente \cite{daoliang1}.\\
 
Cuando la separación entre puertas es muy grande, pude ocurrir que las mismas queden ubicadas muy cerca de una esquina. Según las 
Refs.~\cite{kirchner1,daoliang1}, este hecho perjudica la eficiencia 
en la evacuación, pero no se conoce una explicación precisa de este 
fenómeno, probablemente esté relacionado con el aumento en la presión que sienten los individuos. \\

\noindent Según algunos investigadores (Ref.~\cite{huanhuan1}, es posible identificar cuatro distancias capaces de afectar la eficiencia de la evacuación de individuos: el ancho total de las salida (es decir, sumar los anchos de cada puerta), la distancia de separación entre puertas, la diferencia de ancho entre las puertas y la distancia a la esquina más cercana \cite{huanhuan1}. \\

Si el ancho total de la suma de las puertas se mantiene constante, la distancia entre puertas parece regular en cierta forma la eficiencia en la evacuación (ver Ref.~\cite{huanhuan1}).


En suma los resultados previos sugieren que:

\begin{enumerate}[(a)]

\item Según Kirchner y Schadschneider~\cite{kirchner1}, el tiempo de evacuación es independiente de la distancia de separación para el caso de peatones con comportamiento individualista. Esto será refutado en la sección \ref{dos_puertas}.

\item Si los peatones se mueven en manada, el tiempo de evacuación para pequeñas separaciones es mayor~\cite{kirchner1}. En este trabajo de tesis no se analiza el caso de movimientos en manada. 

\item Según Perez~\cite{perez1}, el número de peatones que abandonó el recinto por unidad de tiempo mostró una disminución para distancias de separación menores al ancho de cuatro puertas. Esto no concuerda con nuestros resultados discutidos en la sección \ref{dos_puertas}.

\item No importa cuan separadas están las dos puertas, el rendimiento de la evacuación no mejora el doble respecto al rendimiento que tiene una única salida (con el ancho de dos puertas). Se mostrarán resultados que concuerdan con esta observación en la sección \ref{dos_puertas}. 


\end{enumerate}


Nuestra investigación se centra en configuraciones con puertas de igual tamaño. A diferencia de la literatura arriba mencionada, se examina la dinámica de la evacuación a través del modelo de fuerza social. En el capítulo siguiente se encuentra una descripción de dicho modelo. \\

\newpage

\section{Objetivos}

En este trabajo nos proponemos estudiar la dinámica de evacuación de peatones en estado de pánico en recintos con una y dos salidas disponibles. Se explorarán distintos escenarios con cientos de personas.\\ 

Se compararán campos de presión y velocidad para recintos con puertas anchas y angostas con el fin de comprender cómo el ancho de la salida afecta el comportamiento de la evacuación. La descripción de esta problemática será tratada en la sección \ref{una_puerta}\\

Se modificará la cantidad de peatones y el grado de ansiedad de los individuos para observar cómo estas propiedades alteran la dinámica y encontrar magnitudes que sean independientes de las mismas. Estos aspectos se exhiben a lo largo del capítulo \ref{resultados} \\

Se realizarán simulaciones para saber a qué distancia deben ubicarse las puertas para que la evacuación se efectúe en el menor tiempo posible. Además se buscará responder si una única puerta ancha produce una evacuación igual de eficiente que dos puertas angostas separadas. El abordaje de esta parte de la investigación se desarrolla en la sección \ref{dos_puertas} 




