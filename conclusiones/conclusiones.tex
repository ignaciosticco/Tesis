En este trabajo de tesis se implementaron códigos que permitieron realizar simulaciones de multitudes evacuando en estado de pánico. \\

Para recintos con una única salida se obtuvieron dinámicas diferentes según el tamaño de la puerta. Si esta es ancha el flujo de evacuación es permanente (\emph{i.e.} no hay momentos en los que todos los peatones se encuentren sin poder moverse). Las zonas de mayor presión se dan a los costados de la puerta. Cuando la salida es angosta la evacuación es intermitente (\emph{i.e.} por momentos algunos peatones logran salir y en otras ocasiones todos están quietos), este efecto de Stop-and-go hace que la zona de máxima presión se de en el medio del bulk (dos metros antes de la puerta). 
Para ambos recintos la mayor velocidad de los individuos se tiene en el medio. \\
En cuanto a los recintos con dos puertas en un mismo flanco, se obtuvo  una separación crítica para la cual cambia la pendiente de la curva tiempo de evacuación vs gap. Este valor es $g_c=1,5$~m y coincide aproximadamente con el ancho de dos peatones. Esta cantidad es independiente del número de individuos y el grado de apuro que tengan por salir. Esto permitió afirmar que el $g_c$ afecta a los bloqueos que ocurren en las cercanías de la salida. \\

La forma funcional de la probabilidad de formar small blocking clusters es similar a la forma funcional del tiempo de evacuación, esto denota que los bloqueos de cada una de las puertas son determinantes en la eficiencia de la evacuación. \\

Aumentar el grado de apuro de los individuos mediante la velocidad de deseo genera evacuaciones más lentas así como aumentar la cantidad de individuos en el recinto. Esto condice con la expresión analítica de la presión y los resultados obtenidos de la presión que soportan los individuos del blocking cluster. En todos los casos aumentar $V_d$ o N genera un incremento de la presión y tiene como consecuencia evacuaciones menos eficientes.\\

El mejor rendimiento en las evacuaciones se dio cuando las puertas no tienen separación entre si ($g=0$~m). Separar la salida en dos puertas empeora el rendimiento a pesar de que la apertura total tenga el mismo tamaño.   
