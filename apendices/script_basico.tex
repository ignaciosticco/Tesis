\label{appendix:script}

A modo de ejemplo se describirá un script de Lammps que se hizo para simular 225 individuos en un recinto de 20$\times$20~m con una puerta de 1,2~m. Este programa cuenta con dos ciclos {\tt for}, uno recorre diferentes valores de velocidad de deseo y el otro (anidado) sirve para tener 30 simulaciones con diferentes velocidades iniciales. \\

El programa devuelve un video de la simulación en formato mpg y un archivo txt con una tabla que indica: $v_d$, número de iteración y el tiempo para el cual evacuaron más de 159 individuos.
El script está dividido en cinco partes: condiciones iniciales, fuerzas, evacuados, visualización y ejecución del proceso.\\ 

En las condiciones iniciales se fija la dimensión del problema (2d), las condiciones de contorno (no periódicas en las coordenadas x e y), el sistema de unidades usado (sistema internacional), también se configuran los atributos de los individuos como el tamaño (60 cm diámetro) y su peso (70~kg). La velocidad inicial se fijó con la función {\tt velocity}, la misma genera una distribución gausiana de velocidades que es asignada de forma aleatoria a los individuos. El módulo de la velocidad es 1,7~m/s. La semilla que genera la distribución de velocidades es el número de iteración.\\

En cuanto al tamaño de la región de simulación, se crearon tres zonas: la zona 1 es la que inicialmente posee a los individuos (ordenados tipo red cuadrada separados 1,3~m entre si), la zona 2 es la región de evacuación y la zona 3 es la conexión entre ambas regiones (la puerta por donde pasan).\\

En la parte de fuerzas se especifican las interacciones que sienten los individuos y sus respectivos parámetros. La repulsión entre individuos (pair style social) utiliza la expresión ($\ref{fsocial}$) con los parámetros: A=2000~N, B=0,08~m y se asignó un cutoff de 3,5~m. La fuerza granular (pair style gran/social) se basa en la fórmula ($\ref{frozamiento}$) con $\kappa =2,4 \times 10^5$~kgm$^{-1}$s$^{-1}$. Se usaron los mismos parámetros para las expresiones análogas en la interacción individuo-pared (fix wall/region social y granular). La fuerza de deseo (fix social/self) tomó como parámetro a la velocidad de deseo, ésta varía según el loop del encabezado del programa. \\

En la parte de evacuación se cuenta a todos los individuos que salieron del recinto, es decir, a todos los peatones cuya posición en la coordenada x sea mayor a 20 m. Este cálculo queda almacenado en la variable {\tt evacuados} que se utiliza para terminar la simulación cuando la cantidad de individuos evacuados sea superior a 159. \\
La visualización devuelve un video {\tt in.evacuation.mpg} que permite observar la dinámica de la evacuación. \\

La última parte del programa está destinada a la ejecución del proceso. Calcula nuevas posiciones y velocidades usando el algoritmo de Verlet a través de la función {\tt fix nve}. El timestep utilizado es 0,0001 s. La función {\tt run} 500 aplica 500 iteraciones de Verlet. En la variable {\tt t} se calcula y guarda el tiempo. Una vez que se supera la cantidad de individuos requerida, la función {\tt print} escribe en un txt las siguientes variables: velocidad de deseo, número de iteración y tiempo de evacuación (el tiempo que tardan evacuar más de 159 peatones).\\

En la sintaxis de Lammps, todo texto seguido de un numeral es comentario. A continuación se muestra el código arriba descripto.

\begin{verbatim}
## 2D Panic Evacuation: 225 individuals, room 20x20
#	Loop velocidad de deseo
variable vdmax equal 6
variable vd loop 1 ${vdmax}
label start_of_loop1
#	Loop de iteracion
variable itermax equal 30
variable iter loop 1 ${itermax}
label start_of_loop2

#           INITIAL CONDITIONS
dimension        2
boundary         f f p
units            si
atom_style       sphere
lattice          sq 1.3 origin 0.5 0.5 0.0
region           zona1 block 0 20 0 20 -1 1 units box
region           zona2 block 20.12 40 0 20 -1 1 units box
region           zona3 block 19 21 9.4 10.6 -1 1 units box
region           todas union 3 zona1 zona2 zona3
create_box       1 todas
create_atoms     1 region zona1
set              atom * mass 70.0
set              atom * diameter 0.6
velocity         all create 1e23 ${iter} dist gaussian
comm_modify      vel yes

#           FORCES
pair_style  hybrid/overlay gran/social 0 0 0 240000 0 1 social 0.08 3.5
pair_coeff  * * social 2000 3.5 0.3
pair_coeff  * * gran/social
fix  walls  all wall/region todas social 2000 0.08 3.5
fix  wallg  all wall/region todas granular 240000 120000 0.001    
fix  target all social/self 70 ${vd} xy          

#           EVACUATED
compute     1 all property/atom x
variable    out atom c_1>20.0
compute     mycompute all reduce sum v_out
variable    evacuados equal c_mycompute

#           VISUALIZE
dump  3 all movie 200 in.evacuation.mpg type type &
      axes yes 0.8 0.02 view 0 0 zoom 2 adiam 0.6

#           RUN THE PROCESS
atom_modify   sort 0 0.0
timestep      0.0001
fix           1 all nve
thermo_style  custom step c_mycompute

#	Loop de proceso
variable  nmax equal 20000
variable  n loop ${nmax}
label     start_of_loop3
run       500
if        "${evacuados} > 159" then "jump SELF break"
variable  t equal 0.05*$n
next n
jump SELF start_of_loop3
#	Terminación del proceso
label break
print "${vd}  ${iter}  $t" append print.txt
clear
variable n delete
next iter
jump SELF start_of_loop2
#	Terminacion loop de itercion
clear
next vd
jump SELF start_of_loop1
#	Terminación loop velocidad de deseo
\end{verbatim}
